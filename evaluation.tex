\section{Evaluation}
\label{sec:validation}

Defining the set of algorithms that should be implemented by the
platform and defining the API is beyond the scope of this position
paper.  Instead, this section provides evidence to support our
conjecture.  We use trace-based simulation to evaluate the performance
of two simple classifiers built using common pre-processing algorithms
for two accelerometer-based applications.


\subsection{Applications}

We developed two applications that detect activities performed by 
human subjects:

\begin{enumerate}
\setlength{\itemsep}{-3pt}  

\item A {\em Step Detector} based on the human step detection algorithm
  proposed by Ryan Libby in~\cite{libbyFootstepDetection}. The
  application reduces noise in the sensor data by applying a low-pass 
  filter, computes the magnitude of each acceleration vector and searches 
  for local maxima in the acceleration magnitude values. Local maxima
  in a particular acceleration range are detected as steps, given
  that no other steps were detected within the last 200 milliseconds.

\item A {\em Fall Detector} based on a simple thresholding algorithm,
  as described in~\cite{kangasFallDetection}.  It checks for acceleration magnitude 
  below a small threshold (representing free-fall), followed by a spike in acceleration magnitude 
  above a different threshold (representing the moment of impact).

\end{enumerate}

\subsection{Configurations}

We evaluate the recall and precision of step and fall detection under the following configurations:

\begin{itemize}

\item {\bf Duty Cycling} The applications wake-up at fixed time intervals to collect sensor data for 5 seconds and run the event detection algorithms.

\item {\bf Batching} The applications wake-up at fixed time intervals to receive the batch of sensor data and run the event detection algorithms.

\item {\bf Moto X Step Detection} The built-in step detector is used to wake up the device and run the applications, be it step detection or fall detection.

\item {\bf Moto X Significant Motion} The built-in significant motion detector is used to wake up the device and run the applications.

\item {\bf Simple Configurable Classifier} We constructed simple event classifiers to wake up the device and run the event detection applications. A more detailed description of the 
  classifiers is available in Section~\ref{sec:classifiers}.
  
\end{itemize}

\begin{figure}[t]
\centering
\begin{tikzpicture}
\begin{axis}[
	ybar=0pt,% space of 0pt between adjacent bars
	enlargelimits=0.15,
	legend style={at={(0.5,1.15)},anchor=north,legend columns=-1},
	symbolic x coords={
		Duty Cycling 2 sec,
		Duty Cycling 5 sec,
		Duty Cycling 10 sec,
		Batching 10 sec,
		MotoX Step Detection,
		MotoX Sig. Motion,
		Simple Classifier},
	x tick label style={rotate=45,anchor=east},
	xtick=data,
	%nodes near coords,
	nodes near coords align={vertical},
]
\addplot coordinates {
	(Duty Cycling 2 sec,0.73) 
	(Duty Cycling 5 sec,0.44) 
	(Duty Cycling 10 sec,0.31)
	(Batching 10 sec,1) 
	(MotoX Step Detection, 1.0)
	(MotoX Sig. Motion, 0.95)
	(Simple Classifier, 1.0)};
\addplot +[postaction={pattern=north west lines, pattern color=red!75}] coordinates {
	(Duty Cycling 2 sec,0.476667) 
	(Duty Cycling 5 sec,0.833333) 
	(Duty Cycling 10 sec,1.111111) 
	(Batching 10 sec,1.333333) 
	(MotoX Step Detection, 0.683)
	(MotoX Sig. Motion, 0.883333)
	(Simple Classifier, 0.916667)};
\draw [black, dashed] ({rel axis cs:0,0}|-{axis cs:Simple Classifier,1}) -- ({rel axis cs:1,0}|-{axis cs:Simple Classifier,1}) node [pos=0.2, above] {Oracle};
\legend{Recall,Sleep}
\end{axis}
\end{tikzpicture}
\caption{Footstep Detection}
\label{table:footstepDetectionRecallSleepTime}
\end{figure}


\begin{figure}[t]
\centering
\begin{tikzpicture}
\begin{axis}[
	ybar=0pt,% space of 0pt between adjacent bars
	enlargelimits=0.15,
	legend style={at={(0.5,1.15)},anchor=north,legend columns=-1},
	symbolic x coords={
		Duty Cycling 2 sec,
		Duty Cycling 5 sec,
		Duty Cycling 10 sec,
		Batching 10 sec,
		MotoX Step Detection,
		MotoX Sig. Motion,
		Simple Classifier},
	x tick label style={rotate=45,anchor=east},
	xtick=data,
	%nodes near coords,
	nodes near coords align={vertical},
]
\addplot coordinates {
	(Duty Cycling 2 sec,0.7778) 
	(Duty Cycling 5 sec,0.4444) 
	(Duty Cycling 10 sec,0.3333) 
	(Batching 10 sec,1) 
	(MotoX Step Detection,0.6667)
	(MotoX Sig. Motion, 0.7778)
	(Simple Classifier, 1)};
\addplot +[postaction={pattern=north west lines, pattern color=red!75}] coordinates {
	(Duty Cycling 2 sec,0.291545) 
	(Duty Cycling 5 sec,0.510204) 
	(Duty Cycling 10 sec,0.680272) 
	(Batching 10 sec,0.816327) 
	(MotoX Step Detection, 0.418367)
	(MotoX Sig. Motion, 0.540816)
	(Simple Classifier, 1)};
\draw [black, dashed] ({rel axis cs:0,0}|-{axis cs:Simple Classifier,1}) -- ({rel axis cs:1,0}|-{axis cs:Simple Classifier,1}) node [pos=0.2, above] {Oracle};
\legend{Recall,Sleep}
\end{axis}
\end{tikzpicture}
\caption{Fall Detection}
\label{table:fallDetectionRecallSleepTime}
\end{figure}

\subsection{Simple Configurable Classifiers}
\label{sec:classifiers}

We created simple classifiers for our step and fall detection
applications. We assume the following simple algorithms are supported
by the platform:

\begin{itemize}

\item {\bf Data Cleaning} Noise-reduction algorithm based on exponential moving average with 
  configurable degree of weighting decrease.

\item {\bf Feature Extraction} Computation of magnitude of acceleration vector.

\item {\bf Admission Control} Configurable high or low thresholds on values of extracted features.
  
\end{itemize}

The simple classifiers are simplified versions of the detection algorithms implemented
by the applications.  The {\em step detection classifier} consists of an aggressive noise 
reduction step, extraction of magnitude of the acceleration vectors and checking if the 
acceleration magnitude exceed a threshold. The {\em fall detection classifier} comprises a 
moderate noise reduction step, extraction of magnitude of the acceleration vectors and testing
if the acceleration magnitude falls below a threshold.

\subsection{Traces}

We collected two traces from human subjects using a Motorola Moto X
device, running Android 4.4.4.  The smartphone ran an application that
kept the device always awake and continuously recorded all raw
accelerometer readings as well as {\em step} and {\em significant motion} 
events, which are supported natively by the Moto X.

The first trace is 10 minutes long and consists of readings taken from
a person walking, standing idle, falling to the ground, and performing
other activities (e.g., sitting down, standing up, taking the device
out of the pocket).  This trace was carefully engineered so that
walking and falling account for 40\% and 2\% of the trace,
respectively.  This was done to allow us to evaluate the performance
of applications that track frequent, as well as infrequent, events.  We
collected this trace in a controlled environment which enabled us to
annotate it with high quality ground truth data.

The second trace is 2 hours and 50 minutes long and captures one of
the authors' morning routine (getting ready, having breakfast and
commuting to work).  This trace was not annotated with ground truth
information.  This trace is used to determine the feasibility of 
significant motion events as a wake-up condition for events of
interest that occur infrequently.

\subsection{Results}

Figures~\ref{table:footstepDetectionRecallSleepTime}
and~\ref{table:fallDetectionRecallSleepTime} show the recall and
achieved sleep time for step and fall detection for the ten-minute
trace.  Results are normalized by the recall and sleep time of a
hypothetical ideal implementation that only wakes up when an event of
interest occurs.  The ideal implementation sleeps for 60\% and 98\% of
the time for step detection and fall detection, respectively.

As expected, Duty Cycling is effective at reducing the amount of time
the device is awake (and thus, power consumption), but performs poorly
in terms of event of interest recall.  Batching achieves
good recall, but its sleep performance depends on the frequency of
events of interest.  For infrequent events, such as fall
detection, the batching approach generates a high number 
of wake-ups that result in no events of interest being detected.

Using the built-in step detector to wake up the device and run the step 
and fall detection applications had predictable results, achieving high
recall for step detection and moderate recall for fall detection. However,
it caused numerous unnecessary wake-ups.  Upon inspection, we noticed that 
falls and other events such as sitting, standing and taking the device 
out of the pocket were incorrectly classified as steps.  The lack of 
configuration options available in Android's sensor manager makes it 
impossible to calibrate the built-in step detector for usage by specific 
users.  This highlights the importance of personalizing wake-up
conditions to the user.

When using Android's built-in significant motion events as wake-up 
conditions, the step and fall detection applications experienced high, but
imperfect recall.  To determine the feasibility of significant motion events 
as a wake-up condition for events of interest that occur infrequently, we 
analysed the 2 hour and 50 minute long trace.  We assumed the event of 
interest (e.g., falls) did not occur during the entire length of the 
trace.  For this trace, the significant motion event fired, on average, 
every 2.47 seconds.  Using the significant motion event as a wake-up 
condition for infrequent events of interest would be very inefficient from a 
power consumption perspective.  The device would wake up too
frequently, without actually detecting any events of interest.


Using simple configurable classifiers to wake up the device achieved perfect recall 
for both step and fall detection, and 
over 90\% of the sleep time of an ideal implementation.  The
difference in sleep time between this approach and the ideal scenario 
provides an upper bound on the potential additional benefits of fully
programmable code offloading.  We note that the classifiers we constructed 
are optimized for the person performing the actions.  Therefore, the
results for this approach might be too optimistic.  Simulations on traces 
collected from other users are likely to show a decrease in precision or 
recall.  However, this approach allows applications to personalize their 
wake-up condition to their user by adjusting the configuration parameters 
of the algorithms used to build the classifiers.






 