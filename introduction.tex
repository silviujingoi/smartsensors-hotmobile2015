
\section{\label{sec:Introduction}Introduction}

Today, smartphones and tablets are used primarily to run interactive
foreground applications, such as games and web browsers.  As a result,
current mobile devices are optimized for a use case where applications
are used intermittently during the day, in sessions that last for
several minutes.  For example, during her break Alice picks up
her phone, checks the weather, reads the latest news and plays a
game for a few minutes, and then puts it away to go back to work.  To
maximize battery life, current mobile devices are designed to go
into sleep state for most of the day when they are not supporting such
interactive usage.

Unfortunately, most mobile platforms are a poor match for a growing
class of mobile applications that perform continuous background
sensing.  Examples range from context-aware applications~\cite{baldauf2007survey,hong2009context}, such as
medical applications~\cite{hameed2003application,preuveneers2008mobile,tsai2007usability} that improve our
well-being or even save lives using activity recognition (e.g., fall
detection), to applications that use participatory sensing to get a
better understanding of the physical world, such as noise pollution
monitoring~\cite{maisonneuve2009citizen,maisonneuve2009noisetube} or traffic prediction~\cite{hull2006cartel}.  While the processing demands of
these applications are modest most of the time, they require periodic
collection of sensor readings, which prevents the device from going to
sleep for extended periods of time.  As a result, applications that
perform continuous sensing may cause the device's battery to drain
within several hours.

To improve support for continuous sensing applications the research
community has proposed the use of fully programmable heterogeneous
architectures~\cite{reflex,littlerock,turducken}.  In these
approaches, developers partition their applications to offload the
initial stages of the application to a low-power processor (or a
hierarchy of processors).  When the code running on the low power
processor detects the occurrence of an event of interest, it proceeds
to wake up the phone and passes control to the rest of the
application.  While the ability to run custom code on the low-power
hardware provides great flexibility, the significant complexity
inherent in this approach has so far prevented its adoption in
commercial devices.  Instead, smartphone manufacturers, realizing the
potential of sensing applications, have recently incorporated
low-power processors into their architectures, but have limited
application developers to APIs that provide fixed functionality, by
either batching sensor readings or recognizing a small number of
predefined activities that can be used as wake-up
conditions~\cite{androidMotionSensors,coreMotion,motox}.
Unfortunately, batching is inefficient for applications that depend on
infrequent events and is not appropriate for applications that require
crisp response time.  Similarly, an activity recognition API provides
little flexibility with no support for applications interested in
events that are not covered by the set of predefined activities.

This paper proposes {\em Simple Configurable Classifiers} (SCC), a new
approach for continuous mobile sensing that splits the work of
energy-efficient event detection between the platform and the
application developer.  In the new approach, the platform implements
common sensor data pre-processing algorithms (e.g., noise reduction,
feature extraction, admission control) that execute on a low-power
sensor hub, and application developers construct simple 
classifiers for events of interest by selecting among the set of
pre-defined common pre-processing algorithms and tuning their
parameters.  The simple classifiers execute on the
low-power sensor hub and, when events of interest are detected, the
main processor is woken up and the rest of the application code is
invoked.
  
We argue that SCC provides a better
balance between flexibility and ease of deployment than existing
approaches.  The approach empowers developers to create a broad set of
event classifiers that can be used to detect a wide range of
activities.  Simultaneously, these classifiers are significantly
easier to program compared to fully programmable offloading.
Moreover, since the pre-processing algorithms are pre-specified, their
implementations can be optimized (by the device manufacturer) for each
low-power processor, improving application portability between
devices.  

The next section discusses existing approaches to continuous sensing
and their limitations.  Section~\ref{sec:conjecture} introduces SCC as
a new approach to supporting continuous mobile sensing.  This section
also discusses the advantages and challenges of the new approach.
Section~\ref{sec:validation} uses trace-based simulation to illustrate
the benefits of leveraging SCC for two
accelerometer-based applications.  Finally,
Section~\ref{sec:conclusion} conclude the paper.

